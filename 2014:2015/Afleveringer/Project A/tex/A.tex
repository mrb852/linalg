\documentclass[12pt,a4paper]{article}

\setlength{\textwidth}{165mm}
\setlength{\textheight}{240mm}
\setlength{\parindent}{0mm} % S{\aa} meget rykkes ind efter afsnit
\setlength{\parskip}{\parsep}
\setlength{\headheight}{0mm}
\setlength{\headsep}{0mm}
\setlength{\hoffset}{-2.5mm}
\setlength{\voffset}{0mm}
\setlength{\footskip}{15mm}
\setlength{\oddsidemargin}{0mm}
\setlength{\topmargin}{0mm}
\setlength{\evensidemargin}{0mm}

\usepackage{amsmath} % flere matematikkommandoer
\makeatletter
\usepackage[all]{xy}
\usepackage{graphicx}    % For grafik (billederfiler)
\usepackage[T1]{fontenc} % For at blande \textsc{} med \textbf{}
\usepackage[utf8]{inputenc}
\usepackage{amsfonts,amsmath,amssymb}
\usepackage{eucal}
\usepackage[danish]{babel}
\usepackage{enumerate}  
\usepackage{hyperref}
\usepackage{url}
\renewcommand*\env@matrix[1][*\c@MaxMatrixCols c]{%
  \hskip -\arraycolsep
  \let\@ifnextchar\new@ifnextchar
  \array{#1}}
\usepackage{mathptmx}

\usepackage{multirow}
\usepackage[dvipsnames,usenames]{color}
\usepackage{tabularx,colortbl,xcolor}
\definecolor{KU-red}{RGB}{144,26,30} 

\DeclareSymbolFont{usualmathcal}{OMS}{cmsy}{m}{n}
\DeclareSymbolFontAlphabet{\mathcal}{usualmathcal}


\DeclareSymbolFont{letters}{OML}{txmi}{m}{it}

\DeclareMathSymbol{\alpha}{\mathord}{letters}{"0B}
\DeclareMathSymbol{\beta}{\mathord}{letters}{"0C}
\DeclareMathSymbol{\gamma}{\mathord}{letters}{"0D}
\DeclareMathSymbol{\delta}{\mathord}{letters}{"0E}
\DeclareMathSymbol{\epsilon}{\mathord}{letters}{"0F}
\DeclareMathSymbol{\zeta}{\mathord}{letters}{"10}
\DeclareMathSymbol{\eta}{\mathord}{letters}{"11}
\DeclareMathSymbol{\theta}{\mathord}{letters}{"12}
\DeclareMathSymbol{\iota}{\mathord}{letters}{"13}
\DeclareMathSymbol{\kappa}{\mathord}{letters}{"14}
\DeclareMathSymbol{\lambda}{\mathord}{letters}{"15}
\DeclareMathSymbol{\mu}{\mathord}{letters}{"16}
\DeclareMathSymbol{\nu}{\mathord}{letters}{"17}
\DeclareMathSymbol{\xi}{\mathord}{letters}{"18}
\DeclareMathSymbol{\pi}{\mathord}{letters}{"19}
\DeclareMathSymbol{\rho}{\mathord}{letters}{"1A}
\DeclareMathSymbol{\sigma}{\mathord}{letters}{"1B}
\DeclareMathSymbol{\tau}{\mathord}{letters}{"1C}
\DeclareMathSymbol{\upsilon}{\mathord}{letters}{"1D}
\DeclareMathSymbol{\phi}{\mathord}{letters}{"1E}
\DeclareMathSymbol{\chi}{\mathord}{letters}{"1F}
\DeclareMathSymbol{\psi}{\mathord}{letters}{"20}
\DeclareMathSymbol{\omega}{\mathord}{letters}{"21}
\DeclareMathSymbol{\varepsilon}{\mathord}{letters}{"22}
\DeclareMathSymbol{\vartheta}{\mathord}{letters}{"23}
\DeclareMathSymbol{\varpi}{\mathord}{letters}{"24}
\DeclareMathSymbol{\varrho}{\mathord}{letters}{"25}
\DeclareMathSymbol{\varsigma}{\mathord}{letters}{"26}
\DeclareMathSymbol{\varphi}{\mathord}{letters}{"27}

\DeclareMathSymbol{\Gamma}{\mathord}{letters}{"00}
\DeclareMathSymbol{\Delta}{\mathord}{letters}{"01}
\DeclareMathSymbol{\Theta}{\mathord}{letters}{"02}
\DeclareMathSymbol{\Lambda}{\mathord}{letters}{"03}
\DeclareMathSymbol{\Xi}{\mathord}{letters}{"04}
\DeclareMathSymbol{\Pi}{\mathord}{letters}{"05}
\DeclareMathSymbol{\Sigma}{\mathord}{letters}{"06}
\DeclareMathSymbol{\Upsilon}{\mathord}{letters}{"07}
\DeclareMathSymbol{\Phi}{\mathord}{letters}{"08}
\DeclareMathSymbol{\Psi}{\mathord}{letters}{"09}
\DeclareMathSymbol{\Omega}{\mathord}{letters}{"0A}

\DeclareMathSymbol{\upGamma}{\mathalpha}{operators}{"00}
\DeclareMathSymbol{\upDelta}{\mathalpha}{operators}{"01}
\DeclareMathSymbol{\upTheta}{\mathalpha}{operators}{"02}
\DeclareMathSymbol{\upLambda}{\mathalpha}{operators}{"03}
\DeclareMathSymbol{\upXi}{\mathalpha}{operators}{"04}
\DeclareMathSymbol{\upPi}{\mathalpha}{operators}{"05}
\DeclareMathSymbol{\upSigma}{\mathalpha}{operators}{"06}
\DeclareMathSymbol{\upUpsilon}{\mathalpha}{operators}{"07}
\DeclareMathSymbol{\upPhi}{\mathalpha}{operators}{"08}
\DeclareMathSymbol{\upPsi}{\mathalpha}{operators}{"09}
\DeclareMathSymbol{\upOmega}{\mathalpha}{operators}{"0A}



\newcommand{\hhemail}[1]{\textsf{#1}}
\newcommand{\hhurl}[1]{{\color{blue}\url{#1}}}

\begin{document}

\newpage

Navn: Christian Enevoldsen


{\bf Opgave 1 }

% OPGAVE 1 A
\textbf{a)}\\

Totalmatricen: \\

\[
\begin{bmatrix}[ccc|c]
  2 & 3 & -1  &  2 \\
  1 & 1 & 1 & 1 \\
  4 & -1 & a &  4
\end{bmatrix}
\]

\[ r1 \leftrightarrow r2 \]

\[
\begin{bmatrix}[ccc|c]
  1 & 1 & 1 & 1 \\
  2 & 3 & -1  &  2 \\
  4 & -1 & a &  4
\end{bmatrix}
\]

\[ r2 \rightarrow r2 - 2r1 \]

\[
\begin{bmatrix}[ccc|c]
  1 & 1 & 1 & 1 \\
  0 & 1 & -3  &  0 \\
  4 & -1 & a &  4
\end{bmatrix}
\]

\[ r3 \rightarrow r3 - 4r1 \]

\[
\begin{bmatrix}[ccc|c]
  1 & 1 & 1 & 1 \\
  0 & 1 & -3  &  0 \\
  0 & -5 & a-4 &  0
\end{bmatrix}
\]

\[ r3 \rightarrow r3 + 5r2 \]

\[
\begin{bmatrix}[ccc|c]
  1 & 1 & 1 & 1 \\
  0 & 1 & -3  &  0 \\
  0 & 0 & a-19 &  0
\end{bmatrix}
\]

Matricen er på reduceret rækkeechelonform men ikke reduceret da der er flere konstanter i hver række.

% OPGAVE 1 B
\textbf{b)}\\

\[a = 19\]

\[
\begin{bmatrix}[ccc|c]
  1 & 1 & 1 & 1 \\
  0 & 1 & -3  &  0 \\
  0 & 0 & 0 &  0
\end{bmatrix}
\]

\[
\begin{bmatrix}[ccc|c]
  1 & 1 & 1 & 1 \\
  0 & 1 & -3  &  0 \\
  0 & 0 & 0 &  0
\end{bmatrix}
\]

\[
\begin{bmatrix}[ccc|c]
  1 & 0 & 4 & 1 \\
  0 & 1 & -3  &  0 \\
  0 & 0 & 0 &  0
\end{bmatrix}
\]

$x1 = 1 - 4t$\\
$x2 = 3t$\\
$x3 = t$

\textbf{c}\\

\[a = 20\]

\[
\begin{bmatrix}[ccc|ccc]
  1 & 1 & 1 & 1 & 0 & 0 \\
  0 & 1 & -3  &  0 & 1 & 0 \\
  0 & 0 & 1 &  0 & 0 & 1
\end{bmatrix}
\]

\[ r1 \rightarrow r1 - r2 \]

\[
\begin{bmatrix}[ccc|ccc]
  1 & 0 & 4 & 1 & -1 & 0 \\
  0 & 1 & -3  &  0 & 1 & 0 \\
  0 & 0 & 1 &  0 & 0 & 1
\end{bmatrix}
\]

\[ r2 \rightarrow r2 + 3r3 \]

\[
\begin{bmatrix}[ccc|ccc]
  1 & 0 & 4 & 1 & -1 & 0 \\
  0 & 1 & 0  &  0 & 1 & 3 \\
  0 & 0 & 1 &  0 & 0 & 1
\end{bmatrix}
\]

\[ r1 \rightarrow r1 - 4r3 \]

\[
\begin{bmatrix}[ccc|ccc]
  1 & 0 & 0 & 1 & -1 & -4 \\
  0 & 1 & 0  &  0 & 1 & 3 \\
  0 & 0 & 1 &  0 & 0 & 1
\end{bmatrix}
\]

Den inverse er dermed\\

\[ 
\begin{bmatrix}[ccc]
  1 & -1 & -4 \\
  0 & 1 & 3 \\
  0 & 0 & 1
\end{bmatrix}
\]


{\bf Opgave 2 }

% OPGAVE 1 A
\textbf{a)}\\

\[
\textbf{E}_1 =   
\begin{bmatrix}[ccc]
  1 & 0 & 0 \\
  -4 & 1 & 0 \\
  0 & 0 & 1
\end{bmatrix}:
r_2 \rightarrow r_2 -4r_1\]

\[
\textbf{E}_2 =   
\begin{bmatrix}[ccc]
  1 & 0 & 0 \\
  0 & 0 & 1 \\
  0 & 1 & 0
\end{bmatrix}:
r_2 \leftrightarrow r_3\]

\[
\textbf{E}_3 =   
\begin{bmatrix}[ccc]
  1 & 0 & 0 \\
  0 & 1 & 0 \\
  0 & 0 & \frac{1}{5}
\end{bmatrix}:
r_3 \rightarrow \frac{1}{5} r_3\]

\[
\textbf{E}_4 =   
\begin{bmatrix}[ccc]
  1 & 0 & 1 \\
  0 & 1 & 0 \\
  0 & 0 & 1
\end{bmatrix}:
r_1 \rightarrow r_1 + r_3\]

\textbf{B}\\

\[
\textbf{F} = 
\begin{bmatrix}[ccc]
  1 & 0 & 1 \\
  0 & 1 & 0 \\
  0 & 0 & 1
\end{bmatrix} \cdot
\begin{bmatrix}[ccc]
  1 & 0 & 0 \\
  0 & 1 & 0 \\
  0 & 0 & \frac{1}{5} 
\end{bmatrix}\cdot
\begin{bmatrix}[ccc]
  1 & 0 & 0 \\
  0 & 0 & 1 \\
  0 & 1 & 0
\end{bmatrix} \cdot
\begin{bmatrix}[ccc]
  1 & 0 & 0 \\
  -4 & 1 & 0 \\
  0 & 0 & 1
\end{bmatrix} =
\]
\[
\begin{bmatrix}[ccc]
  1 & 0 & \frac{1}{5} \\
  0 & 1 & 0 \\
  0 & 0 & \frac{1}{5}
\end{bmatrix} \cdot 
\begin{bmatrix}[ccc]
  1 & 0 & 0 \\
  0 & 0 & 1 \\
  0 & 1 & 0
\end{bmatrix} \cdot
\begin{bmatrix}[ccc]
  1 & 0 & 0 \\
  -4 & 1 & 0 \\
  0 & 0 & 1
\end{bmatrix} = 
\]

\[
\begin{bmatrix}[ccc]
  1 & \frac{1}{5} & 0 \\
  0 & 0 & 1 \\
  0 & \frac{1}{5} & 0
\end{bmatrix} \cdot
\begin{bmatrix}[ccc]
  1 & 0 & 0 \\
  -4 & 1 & 0 \\
  0 & 0 & 1
\end{bmatrix} = 
\]
\[
\begin{bmatrix}[ccc]
  1 & \frac{1}{5} & 0 \\
  0 & 0 & 1 \\
  0 & \frac{1}{5} & 0
\end{bmatrix} \cdot
\begin{bmatrix}[ccc]
  1 & 0 & 0 \\
  -4 & 1 & 0 \\
  0 & 0 & 1
\end{bmatrix} = 
\begin{bmatrix}[ccc]
  \frac{1}{5} & \frac{1}{5} & 0 \\
  0 & 0 & 1 \\
  -\frac{4}{5} & \frac{1}{5} & 0
\end{bmatrix}
\] 


\[
\textbf{G} = 
\begin{bmatrix}[ccc]
  1 & 0 & 0 \\
  4 & 0 & 0 \\
  0 & 0 & 1
\end{bmatrix} \cdot
\begin{bmatrix}[ccc]
  1 & 0 & 0 \\
  0 & 0 & 1 \\
  0 & 1 & 0
\end{bmatrix} \cdot 
\begin{bmatrix}[ccc]
  1 & 0 & 0 \\
  0 & 1 & 0 \\
  0 & 0 & 5
\end{bmatrix} \cdot
\begin{bmatrix}[ccc]
  1 & 0 & -1 \\
  0 & 1 & 0 \\
  0 & 0 & 1
\end{bmatrix} =
\]
\[ 
\begin{bmatrix}[ccc]
  1 & 0 & 0 \\
  4 & 0 & 0 \\
  0 & 1 & 0
\end{bmatrix} \cdot
\begin{bmatrix}[ccc]
  1 & 0 & 0 \\
  0 & 1 & 0 \\
  0 & 0 & 5
\end{bmatrix} \cdot
\begin{bmatrix}[ccc]
  1 & 0 & -1 \\
  0 & 1 & 0 \\
  0 & 0 & 1
\end{bmatrix} =
\]
\[ 
\begin{bmatrix}[ccc]
  1 & 0 & 0 \\
  4 & 0 & 0 \\
  0 & 1 & 0
\end{bmatrix} \cdot
\begin{bmatrix}[ccc]
  1 & 0 & -1 \\
  0 & 1 & 0 \\
  0 & 0 & 1
\end{bmatrix} =
\begin{bmatrix}[ccc]
  1 & 0 & -1 \\
  4 & 0 & -4 \\
  0 & 0 & 1
\end{bmatrix}
\]

\textbf{C}

$AF = I_m \leftrightarrow A = F^{-1} I_m$\\

$F^{-1} = \begin{bmatrix}[ccc]
1 & 0 & -1\\
4 & 0 & 1\\
0 & 1 & 0
\end{bmatrix}  $\\

\textbf{Opgave 3} \\

\textbf{a}\\

Nabomatricen for grafen ser således ud
\[
\begin{bmatrix}
0 & 1 & 1 & 0 & 1\\
1 & 0 & 0 & 0 & 1 \\
0 & 1 & 0 & 1 & 0 \\
0 & 0 & 1 & 0 & 0 \\
1 & 0 & 0 & 0 & 0
\end{bmatrix}
\]

Det aflæses fra matricen at antallet af veje er 12.

\textbf{b}

For god orden opskrives antallet af udgående links fra webbet.

$N_{web_1} = 3$\\
$N_{web_2} = 2$\\
$N_{web_3} = 2$\\
$N_{web_4} = 1$\\
$N_{web_5} = 1$\\

Linkmatricen opskrives:

\[
\begin{bmatrix}
0 & \frac{1}{5} & 0 & 0 & 1 \\
\frac{1}{3} & 0 & \frac{1}{5} & 0 & 0\\
\frac{1}{3} & 0 & 0 & 1 & 0  \\
0 & 0 & \frac{1}{5} & 0 & 0 \\
\frac{1}{3} & \frac{1}{5} & 0 & 0 & 0  
\end{bmatrix}
\]

\textbf{c}

Ligningssystem opskrives på formlen $\textbf{Ax} = \textbf{x}$

\[
\textbf{A} = \begin{bmatrix}
0 & \frac{1}{5} & 0 & 0 & 1 \\
\frac{1}{3} & 0 & \frac{1}{5} & 0 & 0\\
\frac{1}{3} & 0 & 0 & 1 & 0  \\
0 & 0 & \frac{1}{5} & 0 & 0 \\
\frac{1}{3} & \frac{1}{5} & 0 & 0 & 0  
\end{bmatrix} \text{,} \textbf{x} = 
\begin{bmatrix}
w_1\\w_2\\w_3\\w_4\\w_5
\end{bmatrix}
\]

Vi omskriver systemet til $\textbf{Ax} - \textbf{x} = 0 $
og opstiller den tilsvarende totalmatrix


\[
\textbf{A} = \begin{bmatrix}[ccccc|c]
0 & \frac{1}{5} & 0 & 0 & 1 & 0 \\
\frac{1}{3} & 0 & \frac{1}{5} & 0 & 0 & 0\\
\frac{1}{3} & 0 & 0 & 1 & 0  & 0\\
0 & 0 & \frac{1}{5} & 0 & 0 & 0\\
\frac{1}{3} & \frac{1}{5} & 0 & 0 & 0 & 0  
\end{bmatrix}
\]

Vi laver rækkeoperationer og får identitetsmatricen, hvilket betyder at der alle sider er lige vigtige (nok en fejl et sted)



\textbf{Opgave 4} \\
Se koden i "src/".

Eksempel på kørsel

A:\\

\text{[1.0 4.0 7.0]}\\

\text{[2.0 5.0 8.0]}\\
\text{[3.0 6.0 9.0]}\\

B: \\
\text{[-9.0 6.0 -3.0]}\\
\text{[8.0 -5.0 2.0]}\\
\text{[-7.0 4.0 -1.0]}\\

A new: \\
\text{[1.0 4.0 -4.0]}\\
\text{[2.0 5.0 8.0]}\\
\text{[3.0 6.0 9.0]}\\

A after mul and add: \\
\text{[-7.0 14.0 11.0]}\\
\text{[12.0 5.0 18.0]}\\
\text{[-1.0 16.0 17.0]}\\

\end{document}

