\documentclass[12pt]{article}
\usepackage{amsmath} % flere matematikkommandoer
\makeatletter
\renewcommand*\env@matrix[1][*\c@MaxMatrixCols c]{%
  \hskip -\arraycolsep
  \let\@ifnextchar\new@ifnextchar
  \array{#1}}
\makeatother
\usepackage[utf8]{inputenc} % æøå
\usepackage[T1]{fontenc} % mere æøå
\usepackage[danish]{babel} % orddeling
\usepackage{verbatim} % så man kan skrive ren tekst
\usepackage[all]{xy} % den sidste (avancerede) formel i dokumentet

\begin{titlepage}

\title{Projekt A}
\author{Christian Hohlmann Enevoldsen, MRB852, Hold 2}
\begin{document}
\maketitle

\end{titlepage}

{\setlength{\parindent}{0 cm}

\textbf{\large Opgave 1.}\\

\textbf{a)}\\

Totalmatricen: \\

$
\begin{bmatrix}[ccc|c]
  2 & 3 & 1  &  11 \\
  1 & 1 & 1 & 6 \\
  5 & -1 & a &  36
\end{bmatrix}
$
$r_1 - 2r_2 \rightarrow r_1$
$
\begin{bmatrix}[ccc|c]
  0 & 1 & -1  &  -1 \\
  1 & 1 & 1 & 6 \\
  5 & -1 & a &  36
\end{bmatrix}
$
$r_3 - 5r_2 \rightarrow r_3$
$
\begin{bmatrix}[ccc|c]
  0 & 1 & -1  &  -1 \\
  1 & 1 & 1 & 6 \\
  0 & -6 & a - 5 &  6
\end{bmatrix}
$
\\\\\\
$r_3 + 6r_1 \rightarrow r_3$
$
\begin{bmatrix}[ccc|c]
  0 & 1 & -1  &  -1 \\
  1 & 1 & 1 & 6 \\
  0 & 0 & a - 11 &  0
\end{bmatrix}
$
$r_1 \leftrightarrow r_2$
$
\begin{bmatrix}[ccc|c]
  
  1 & 1 & 1 & 6 \\
0 & 1 & -1  &  -1 \\  
0 & 0 & a - 11 &  0
\end{bmatrix}
$
\\\\

Matricen er på rækkeechelonform, da alle omdrejningspunkterne (eng.: pivots) er stærkt til højre for omdrejningspunkterne i rækken ovenfor og alle elementer under omdrejningspunket i kolonerne er nuller.  Det er ikke på reduceret rækkeechelonform, fordi omdrejningspunkterne ikke er de eneste ikke nuller i deres kolonne.\\

\textbf{b)}\\

$
\begin{bmatrix}[ccc|c]
  
1 & 1 & 1 & 6 \\
0 & 1 & -1  &  -1 \\  
0 & 0 & 0 &  0
\end{bmatrix}
$
$r_1 - r_2 \rightarrow r_1$
$
\begin{bmatrix}[ccc|c]
  1 & 0 & 2 & 7 \\
0 & 1 & -1  &  -1 \\  
0 & 0 & 0 &  0
\end{bmatrix}
$\\\\

Det kan ses at $x_3$ må være en fri variable.

$x_3 = t$

Løsningen er således: $(x_1, x_2, x_3) = (7- 2t, t - 1, t)$\\

\textbf{c)}\\
Koefficientmatrix: 
$
\begin{bmatrix}
2 & 3 & 1  \\
1 & 1 & 1  \\  
5 & -1 & 12
\end{bmatrix}
$\\\\

Totalmatricen opskrives\\

$
\begin{bmatrix}[ccc|ccc]
2 & 3 & 1 & 1 & 0 & 0 \\
1 & 1 & 1  & 0 & 1 & 0 \\  
5 & -1 & 12 & 0 & 0 & 1
\end{bmatrix}
$\\\\

$r_1 \leftrightarrow r_2$ -> for at sortere\\

$
\begin{bmatrix}[ccc|ccc]
1 & 1 & 1  & 0 & 1 & 0 \\  
2 & 3 & 1 & 1 & 0 & 0 \\
5 & -1 & 12 & 0 & 0 & 1
\end{bmatrix}
$
$$r_2 - 2r_1 \rightarrow r_2$$
$$r_3 - 5r_1 \rightarrow r_3$$
$
\begin{bmatrix}[ccc|ccc]
1 & 1 & 1  & 0 & 1 & 0 \\  
0 & 1 & -1 & 1 & -2 & 0 \\
0 & -6 & 7 & 0 & -5 & 1
\end{bmatrix}
$

$$r_2 - r_3 \rightarrow r_2$$
$$r_1 - r_3 \rightarrow r_1$$
$
\begin{bmatrix}[ccc|ccc]
1 & 1 & 0  & -6 & 18 & -1 \\  
0 & 1 & 0 & 7 & -19 & 1 \\
0 & 0 & 1 & 6 & -17 & 1
\end{bmatrix}
$\\


$$r_1 - r_2 \rightarrow r_1$$
$
\begin{bmatrix}[ccc|ccc]
1 & 0 & 0  & -13 & 37 & -2 \\  
0 & 1 & 0 & 7 & -19 & 1 \\
0 & 0 & 1 & 6 & -17 & 1
\end{bmatrix}
$\\\\

\textbf{\large Opgave 2.}\\

\textbf{a)}\\

\textbf{$E_1$} = $
\begin{bmatrix}
1 & 0 & 0 \\
0 & 1 & 0 \\
-3 & 0 & 1
\end{bmatrix}
$
\textbf{$E_2$} = $
\begin{bmatrix}
1 & 0 & 0 \\
0 & 0 & 1 \\
0 & 1 & 0
\end{bmatrix}
$
\textbf{$E_3$} = $
\begin{bmatrix}
1 & 0 & 0 \\
0 & 1 & 0 \\
0 & 0 & 0,5
\end{bmatrix}
$
\textbf{$E_4$} = $
\begin{bmatrix}
1 & 0 & -1 \\
0 & 1 & 0 \\
0 & 0 & 1
\end{bmatrix}
$\\

\textbf{b)}\\

$$\textbf{F} =  
\begin{bmatrix}
1 & 0 & -1 \\
0 & 1 & 0 \\
0 & 0 & 1
\end{bmatrix}
\cdot
\begin{bmatrix}
1 & 0 & 0 \\
0 & 1 & 0 \\
0 & 0 & 0,5
\end{bmatrix}
\cdot 
\begin{bmatrix}
1 & 0 & 0 \\
0 & 0 & 1 \\
0 & 1 & 0
\end{bmatrix}
\begin{bmatrix}
1 & 0 & 0 \\
0 & 1 & 0 \\
-3 & 0 & 1
\end{bmatrix}
$$

$$=
\begin{bmatrix}
1 & 0 & -0,5\\
0 & 1 & 0\\
0 & 0 & 0,5
\end{bmatrix}
\cdot
\begin{bmatrix}
1 & 0 & 0 \\
0 & 0 & 1 \\
0 & 1 & 0
\end{bmatrix}
\cdot
\begin{bmatrix}
1 & 0 & 0 \\
0 & 1 & 0 \\
-3 & 0 & 1
\end{bmatrix}
=$$
$$
\begin{bmatrix}
1 & -0,5 & 0 \\
0 & 0 & 1 \\
0 & 0,5 & 0
\end{bmatrix}
\cdot
\begin{bmatrix}
1 & 0 & 0 \\
0 & 1 & 0 \\
-3 & 0 & 1
\end{bmatrix} =
\begin{bmatrix}
1 & -0,5 & 0 \\
-3 & 0 & 1 \\
0 & 0,5 & 0
\end{bmatrix}
$$

\textbf{$E_1^{-1}$} = $r_3 + 3r_1 \rightarrow r_3$ = 
$\begin{bmatrix}
1 & 0 & 0 \\
0 & 1 & 0 \\
3 & 0 & 1
\end{bmatrix}$\\\\


\textbf{$E_2^{-1}$} = $r_3 \leftrightarrow r_2 $ = 
$\begin{bmatrix}
1 & 0 & 0 \\
0 & 0 & 1 \\
0 & 1 & 0
\end{bmatrix}$\\\\


\textbf{$E_3^{-1}$} = $2r_3 \rightarrow r_3 $ = 
$\begin{bmatrix}
1 & 0 & 0 \\
0 & 1 & 0 \\
0 & 0 & 2
\end{bmatrix}$\\\\


\textbf{$E_4^{-1}$} = $r_1 + r_3 \rightarrow r_1 $ = 
$\begin{bmatrix}
1 & 0 & 1 \\
0 & 1 & 0 \\
0 & 0 & 1
\end{bmatrix}$


$$\textbf{G} = 
\begin{bmatrix}
1 & 0 & 0 \\
0 & 1 & 0 \\
3 & 0 & 1
\end{bmatrix} \cdot
\begin{bmatrix}
1 & 0 & 0 \\
0 & 0 & 1 \\
0 & 1 & 0
\end{bmatrix} \cdot
\begin{bmatrix}
1 & 0 & 0 \\
0 & 1 & 0 \\
0 & 0 & 2
\end{bmatrix} \cdot 
\begin{bmatrix}
1 & 0 & 1 \\
0 & 1 & 0 \\
0 & 0 & 1
\end{bmatrix} = 
$$
$$
\begin{bmatrix}
1 & 0 & 0 \\
0 & 0 & 1 \\
3 & 1 & 0
\end{bmatrix}\cdot
\begin{bmatrix}
1 & 0 & 0 \\
0 & 1 & 0 \\
0 & 0 & 2
\end{bmatrix} \cdot 
\begin{bmatrix}
1 & 0 & 1 \\
0 & 1 & 0 \\
0 & 0 & 1
\end{bmatrix} = 
$$
$$
\begin{bmatrix}
1 & 0 & 0 \\
0 & 0 & 2 \\
3 & 1 & 0
\end{bmatrix} \cdot 
\begin{bmatrix}
1 & 0 & 1 \\
0 & 1 & 0 \\
0 & 0 & 1
\end{bmatrix} = 
\begin{bmatrix}
1 & 0 & 1 \\
0 & 0 & 2\\
3 & 1 & 3
\end{bmatrix}
$$\\

\textbf{c)}\\

$AF = I_m \leftrightarrow A = F^{-1} I_m$\\

Vi finder $F^{-1}$\\
$$
\begin{bmatrix}[ccc|ccc]
1  & -0,5 & 0 & 1 & 0 & 0 \\
-3 & 0 & 1 & 0 & 1 & 0 \\
0 & 0,5 & 0 & 0 & 0 & 1
\end{bmatrix}
$$
$$ r_3 \leftrightarrow r_2$$
$$ r_3 + 3r_1$$
$$r_3 + 3r_2$$
$$2r_2 \rightarrow r_2$$
$$r_1 + 0,5r_2$$
$$
\begin{bmatrix}[ccc|ccc]
1  & 0 & 0 & 1 & 0 & 1 \\
0 & 1 & 0 & 0 & 0 & 2\\
0 & 0 & 1 & 3 & 1 & 3 
\end{bmatrix}
$$

Det ses således at $F^{-1} = G = A$ fordi identitetsmatrixen gange  invers er lig med sig selv. \\

\textbf{\large Opgave 3.}\\

\textbf{a)}\\
$$
\textbf{N} = 
\begin{bmatrix}
0 & 1 & 1 & 0 & 1 \\
0 & 0 & 0 & 0 & 1 \\
0 & 1 & 0 & 1 & 0 \\
0 & 1 & 1 & 0 & 0 \\
1 & 0 & 0 & 0 & 0
\end{bmatrix}
$$

Det er let at se i det givet at der er 13 veje, da der i $N_{15}$ står 13.\\

\textbf{b)}\\

$$
\textbf{A} = 
\begin{bmatrix}
0 & 0 & 1 & 1 \over 2 \\
1 \over 3 & 0 & 0 & 0 \\
1 \over 3 & 1 \over 2 & 0 & 1 \over 2 \\
1 \over 3 & 1 \over 2 & 0 & 0 
\end{bmatrix}
$$


\textbf{c)}\\

For at finde $x$  sættes ligningssystemet op i en totalmatrix

$$
\begin{bmatrix}[cccc|c]
0 & 0 & 1 & 1 \over 2 & x_1\\
1 \over 3 & 0 & 0 & 0 & x_2 \\
1 \over 3 & 1 \over 2 & 0 & 1 \over 2 & x_3 \\
1 \over 3 & 1 \over 2 & 0 & 0 & x_4
\end{bmatrix}
$$

Det kan omskrives til totalmatricen $Ax -x = 0$
$$
\begin{bmatrix}[cccc|c]
-1 & 0 & 1 & 1 \over 2 & 0\\
1 \over 3 & -1 & 0 & 0 & 0 \\
1 \over 3 & 1 \over 2 & -1 & 1 \over 2 & 0 \\
1 \over 3 & 1 \over 2 & 0 & -1 & 0
\end{bmatrix}
$$

System løses med rækkeoperationer og vi får:

$$
\begin{bmatrix}
x_1 \\
x_2\\
x_3\\
x_4
\end{bmatrix}
=
\begin{bmatrix}
2t \\
{2 \over 3} t\\
{3 \over 2} t\\
t
\end{bmatrix}$$

rangordningen må således være: $x_1, x_3, x_4, x_2$ hvor $x_1$ er vigtigst.\\

\textbf{\large Opgave 4.}\\

\textbf{a)}\\

Opgave4a.java løser opgaven.

Ved at lave et en løkke der kører fra 1 til og med 3 inde i en ækvivalent løkke er det let
at gennemløbe koordinaterne i matrix A. Det væsentlige er variablen c, som bliver klargjort i 
første løkke og gennemløbende bliver forøget med 1 i hver iteration i den indlejrede løkke.\\

Samme teknik bruges til at lave matrix B. Her starter vi dog med at initialisere variablen d med 9
og derefter formindskes d med 1 i den indlejrede løkke. Dog vil den ved uligetegn tildele d værdien 
$-d + 1$ så den forøges i stedet. Dette er ikke en fejl, da vi så skifter fortegn inden d bruges.

For at ændre eller tildele værdier i matrix bruges funktionen\\ set$(int$ $m$, $int $ $n$$)$\\

Add metoden bruges med en instans. f.eks. A.add(matrix:B), og returnerer en ny matrix. Derfor er det vigtigt at man ikke regner med at A ændres.\\


matrix.mul(matrix:B) implementerede jeg ved at bruge B.transpose()  så jeg udelukkende kunne 
arbejde med søjler eller rækker uden forvirring. Dernæst har jeg via 3 for løkker formået at multiplicere B's søjler med A's rækker, samtidig med jeg bruger jeg indekset for det element som bliver brugt i det øjeblik til at bestemme indekset for summen af regnestykket.

Opgave4c.java viser hvordan man bruger mul(matrix:B).\\

\textbf{Eksempel på kørsel af Opgave4ajava}\\

Printer Matrix A.

[1,000000 2,000000 3,000000]

[4,000000 5,000000 6,000000]

[7,000000 8,000000 9,000000]\\

Printer Matrix B.

[-9,000000 8,000000 -7,000000]

[6,000000 -5,000000 4,000000]

[-3,000000 2,000000 -1,000000]\\

Ændrer $a_{12} $ til 4\\

Printer$ A_{12}.$\\

$A_{12}$: 4,000000\\

Prøver at sætte $A_{44}$ = 5. Fejl printes\\

Elements outside the matrix cannot be assigned values\\

Printer A + B

[-8,000000 12,000000 -4,000000]

[10,000000 0,000000 10,000000]

[4,000000 10,000000 8,000000]\\

\textbf{Eksempel på kørsel af Opgave4c.java}\\

Udskriver AB

[-6,000000 4,000000 -2,000000]

[-24,000000 19,000000 -14,000000]

[-42,000000 34,000000 -26,000000]\\

Udskriver BA

[-26,000000 -34,000000 -42,000000]

[14,000000 19,000000 24,000000]

[-2,000000 -4,000000 -6,000000]


}
\end{document}

