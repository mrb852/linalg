\documentclass[12pt]{article}
\usepackage{amsmath} % flere matematikkommandoer
\makeatletter
\renewcommand*\env@matrix[1][*\c@MaxMatrixCols c]{%
  \hskip -\arraycolsep
  \let\@ifnextchar\new@ifnextchar
  \array{#1}}
\makeatother
\usepackage[utf8]{inputenc} % æøå
\usepackage[T1]{fontenc} % mere æøå
\usepackage[danish]{babel} % orddeling
\usepackage{verbatim} % så man kan skrive ren tekst
\usepackage[all]{xy} % den sidste (avancerede) formel i dokumentet

\begin{titlepage}

\title{Projekt C}
\author{Christian Hohlmann Enevoldsen, MRB852, Hold 2}
\begin{document}
\maketitle

\end{titlepage}

{\setlength{\parindent}{0 cm}

\textbf{\large Opgave 1.}\\

\textbf{a)}\\

\[
\textbf{u}_1 = \left ( \begin{matrix} 2 \\ 1 \\ -2 \\ 0 \end{matrix} \right), %u1
\textbf{u}_2 = \left ( \begin{matrix} 2 \\ 0 \\ 2 \\ -1 \end{matrix} \right )\\\\%u2
\]

\[
%Vis orthogonal
\textbf{u}_1 \perp \textbf{u}_2 \text{ hvis }   \textbf{u}_1 \cdot \textbf{u}_2 = 0
\]

%beregning
\begin{align*}
\textbf{u}_1 \cdot \textbf{u}_2 &= 2 \cdot 2 + 1 \cdot 0  - 2 \cdot 2  - 1 \cdot 0 \\
&= 4 - 4\\ 
&= 0
\end{align*}

Da $u_3$ kan skrives som en lineær kombination af $u_1$ og $u_2$ er den ikke en del af basen for $\mathcal{U} $

\[
 \text{ergo er } span\{ \textbf{u}_1, \textbf{u}_2 \} \text{  en ortogonal basis for } \mathcal{U} 
\]

%Pv = projU(v)

\textbf{b)}\\

\begin{align*}
A &= span \{  \mathcal{U} \} \\
Pv = proj  \mathcal{U}_v &= {AA^T}_v \\
Pv =
\begin{bmatrix}
2 & 2 \\ 1 & 0 \\ -2 & 2 \\ 0 & -1
\end{bmatrix}
\begin{bmatrix}
2 & 1 & -2 & 0\\ 2 & 0 & 2 & -1
\end{bmatrix}
& = \begin{bmatrix}
 8 &  2 &  0 & -2\\
 2 &  1 & -2 &  0\\
 0 & -2 &  8 & -2\\
-2 &  0 & -2 &  1
\end{bmatrix}v\\
\leftrightarrow P &= \begin{bmatrix}
 8 &  2 &  0 & -2\\
 2 &  1 & -2 &  0\\
 0 & -2 &  8 & -2\\
-2 &  0 & -2 &  1
\end{bmatrix}
\end{align*}

\textbf{c)}\\
\begin{align*}
Pv &=
\begin{bmatrix}
 8 &  2 &  0 & -2\\
 2 &  1 & -2 &  0\\
 0 & -2 &  8 & -2\\
-2 &  0 & -2 &  1
\end{bmatrix} 
\begin{bmatrix}
0 \\9 \\ 0\\9
\end{bmatrix} = 
\begin{bmatrix}
0 \\ 9 \\-36 \\9
\end{bmatrix}
\end{align*}

\textbf{d)}\\

Vi betragter underrummet $\{ \mathcal{U} \} = \{\textbf{u}_1, \textbf{u}_2, \textbf{u}_3\}$\\

Lad $ \textbf{A} = span(\mathcal{\{U}\})$ \\

Vi ved at $ \mathcal{U}^\perp  = null(\textbf{A}^T)$\\

Vi beregner nu $null (\textbf{A})$

\begin{align*} 
\begin{bmatrix}
2 & 2 & 4\\ 1 & 0 & 1 \\ -2 & 2 & 0 \\ 0 & -1 & -1
\end{bmatrix} 
\begin{bmatrix}
x_1 \\ x_2 \\ x_3
\end{bmatrix} &= 
\begin{bmatrix}
0 \\ 0 \\ 0 \\0
\end{bmatrix}\\
&\updownarrow \\
2x_1 + 2x_2 + 4x_3 &= 0 \\ 
x_1 + x_3 &= 0\\
 -2x_1 + 2x_2 &= 0\\ 
-x_2  -x_3 &= 0\\
&\updownarrow \\
\begin{bmatrix}[ccc|c]
2 & 2 & 4 & 0 \\ 1 & 0 & 1 & 0  \\ -2 & 2 & 0 & 0  \\ 0 & -1 & -1 & 0
\end{bmatrix}  {RREF} &\rightarrow 
\begin{bmatrix}[ccc|c]
1 & 0 & 1 & 0\\
0 & 1 & 1 & 0\\
0 & 0 & 0 & 0\\
0 & 0 & 0 & 0
\end{bmatrix}\\
\begin{bmatrix}
x_1\\x_2\\x_3
\end{bmatrix} &= x_3
\begin{bmatrix}-1 \\ -1 \\ 1 \end{bmatrix} \\\\
null (\textbf{A}) &= span \left( \begin{bmatrix}-1 \\ -1 \\ 1 \end{bmatrix} \right)\\\\
\text{Et basis $\beta$ for }\mathcal{U}^\perp \text{ er derfor } \beta &=
\left\{ \begin{bmatrix}-1 \\ -1 \\ 1 \end{bmatrix} \right\}
\end{align*}

\textbf{\large Opgave 2.}\\

\textbf{a)}\\
\begin{align*}
 q_1 = \frac{v_1}{||v_1||}  &= \begin{bmatrix}1/5\\2/5\\4/5\\2/5\end{bmatrix} \\
y = v_2 -  proj^{ \text{ }  v_2}_{v_1} = 
\begin{bmatrix}-3\\4\\3\\4\end{bmatrix} -
\begin{bmatrix}1\\2\\4\\2\end{bmatrix} &= 
\begin{bmatrix}-4\\2\\-1\\2\end{bmatrix}
\\
q_2 = \frac{y}{||y||} &= \begin{bmatrix}-4/5\\2/5\\-1/5\\2/5\end{bmatrix}
\end{align*}

Den ortonormale basis for  $\mathcal{V}$ er således $ \mathcal{B} = 
\left \{ \begin{bmatrix}1/5\\2/5\\4/5\\2/5\end{bmatrix}, 
 \begin{bmatrix}-4/5\\2/5\\-1/5\\2/5\end{bmatrix}  \right \} $  

\textbf{b)}\\

$[w]_b = \frac{w \cdot q_1}{||q_1||^2} q_1 + ... + \frac{w \cdot q_4}{||q_4||^2} q_4 $\\

Lad $x_n =  \frac{w \cdot q_n}{||q_n||^2} q_n $\\

Længderne på $q_1 ... q_4$ er normaliseret dvs. 1.\\

Derfor er $x_n = w \cdot q_n  $ \\

Og jeg får $(x_1, x_2, x_3, x_4) = (0, 5, 5, 10)$\\

Således\\

 $w = 
0 \begin{bmatrix} 1/5\\2/5\\4/5\\2/5 \end{bmatrix}+
5 \begin{bmatrix} -4/5\\2/5\\-1/5\\2/5 \end{bmatrix}+
5 \begin{bmatrix} 2/5\\-1/5\\-2/5\\4/5 \end{bmatrix}+
10 \begin{bmatrix} -2/5\\-4/5\\2/5\\1/5 \end{bmatrix}$

\textbf{c)}\\

Da vi har med en ortonormal basis at gøre kan man bruge reglen: $Q^{-1} = Q^T$
\[
Q^{-1} =
\begin{bmatrix}
1/5 & -4/5 & 2/5 & -2/5\\
2/5 & 2/5 &-1/5 & -4/5\\
4/5 & -1/5 & -2/5 & 2/5 \\
2/5 & 2/5 & 4/5 & 1/5\\
\end{bmatrix}^T = 
\begin{bmatrix}
1/5 & 2/5 & 4/5 & 2/5 \\
-4/5 & 2/5 & -1/5 & 2/5\\
2/5 & -1/5 & -2/5 & 4/5\\
-2/5 & -4/5 & 2/5 & 1/5
\end{bmatrix} 
\]

\textbf{\large Opgave 4.}\\

\textbf{a, b)}\\

Se Main.java og Matrix.java. Jeg har tilladt mig at ændre navnet til Matrix (fra matrix), da det giver 100 gange mere mening.

Eksempel på kørsel:\\

GramSchmidt\\

[0,666667 -0,333333 -0,533333]

[0,000000 0,000000 0,600000]

[0,333333 -0,666667 0,533333]

[0,666667 0,666667 0,266667]\\

v1...\\

[2,800000]

[0,600000]

[3,200000]

[1,600000]\\

v2...\\

[1,000000]

[2,000000]

[3,000000]

[4,000000]


}
\end{document}

