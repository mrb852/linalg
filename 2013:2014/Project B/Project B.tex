\documentclass[12pt]{article}
\usepackage{amsmath} % flere matematikkommandoer
\makeatletter
\renewcommand*\env@matrix[1][*\c@MaxMatrixCols c]{%
  \hskip -\arraycolsep
  \let\@ifnextchar\new@ifnextchar
  \array{#1}}
\makeatother
\usepackage[utf8]{inputenc} % æøå
\usepackage[T1]{fontenc} % mere æøå
\usepackage[danish]{babel} % orddeling
\usepackage{verbatim} % så man kan skrive ren tekst
\usepackage[all]{xy} % den sidste (avancerede) formel i dokumentet
\begin{titlepage}

\title{Projekt B}
\author{Christian Hohlmann Enevoldsen, MRB852, Hold 2}
\begin{document}
\maketitle

\end{titlepage}

{\setlength{\parindent}{0 cm}

\textbf{\large Opgave 1.}\\

\textbf{a)}\\

Først findes RREF-representation af $A$. Det er heldigvis allerede gjort.
Ud fra det findes de uafhængige søjler. (Pivot columns) - Disse udgør basisen for søjlerummet i $colA$

\[
colA = 
\left\{
\begin{bmatrix}
2\\
1\\
4\\
0
\end{bmatrix} ,
\begin{bmatrix}
1\\
1\\
1\\
3
\end{bmatrix},
\begin{bmatrix}
1\\
0\\
6\\
0
\end{bmatrix}
\right\}
\]

\textbf{b)}\\

\[null(A*) =   
\begin{bmatrix}
1 & 2 & 0 & 2 & 0\\
0 & 0 & 1 & 1 & 0\\
0 & 0 & 0 & 0 & 1\\
0 & 0 & 0 & 0 & 0
\end{bmatrix}  
\begin{bmatrix}
x_1\\x_2\\x_3\\x_4\\x_5 
\end{bmatrix} = 
\begin{bmatrix}
0\\0\\0\\0\\0 
\end{bmatrix} 
\]

Vi ser at:
\[x_1 = -2x_2 - 2x_4\]
\[x_3 = -x_4\]
\[x_5 = 0\]

\[x_2 \text{ og }  x_4 \text{ er frie variable }\]

\[ null(A*) =    \vec{x} = \begin{bmatrix}
x_1\\x_2\\x_3\\x_4\\x_5 
\end{bmatrix} = 
\begin{bmatrix} 
-2x_2 - 2x_4\\ 1 \\-4_x \\1 \\0
\end{bmatrix} =
x_2
\begin{bmatrix} 
-2\\ 1 \\0 \\0 \\0
\end{bmatrix} + x_4
\begin{bmatrix} 
- 2\\ 0 \\-1 \\1 \\0
\end{bmatrix} 
\]

\textbf{c)}\\

\[dim(ker \text{ } T) = dim( null(A) ) = 2. \rightarrow \text{ fordi der er 2 elementer i basisen for } null(A)\]
\[dim(ran \text{ } T) = dim(col A) = rank A = 3 \rightarrow \text{ fordi, der er 3 pivot søjler i } A \] .

\textbf{d)}\\

Vi sætter totalmatricen op:

\[
\begin{bmatrix}[ccccc|c]
2 & 4 & 1 & 5 & 1 & 2\\
1 & 2 & 1 & 3 & 0 & 0 \\
4 & 8 & 1 & 9 & 6 & 9 \\
0 & 0 & 3 & 3 & 0 & -3
\end{bmatrix}
\]

Den omdannes til RREF

\[
\begin{bmatrix}
1 & 2 & 0 & 2 & 0 & 1\\
0 & 0 & 1 & 1 & 0 & -1 \\
0 & 0 & 0 & 0 & 1 & 1 \\
0 & 0 & 0 & 0 & 0 & 0
\end{bmatrix}
\]

Det ses at 

\[x_1 = 1 - 2x_1 - 2x_4\]
\[x_3 = -1 - x_4\]
\[x_5 = 1\]


Vi ved at første tredje og femte søjle udgør en basis for $col A$
Derfor kan vi bruge dem i et ligningssystem

\[
\begin{bmatrix}[ccc|c]
2 & 1 & 1 & 2\\
1 & 1 & 0 & 0\\
4 & 1 & 6 & 9\\
0 & 3 & 0 & -3
\end{bmatrix}
\]

Der laves rækkeoperationer og vi får

\begin{align*}
r_2 &\leftrightarrow r_1\\
r_2 - 2r_1&\rightarrow r_1\\
r_3 - 4r_1 &\rightarrow r_1\\
r_2 &\rightarrow -r_2\\
r_3 + 3r_2 &\rightarrow r_3\\
r_4 - 3r_2 &\rightarrow r_4\\
r_4 - r_3 &\rightarrow r_4\\
r_3 / 3 &\rightarrow r_3\\
r_2 + r_3 &\rightarrow r_2\\
r_1 + r_2 &\rightarrow r_1\\
\end{align*}

\[
\begin{bmatrix}[ccc|c]
1 & 0 & 0 & 1\\
0 & 1 & 0 & -1\\
0 & 0 & 1 & 1\\
0 & 0 & 0 & 0
\end{bmatrix}
\]

Her vises så at en mulighed kunne være: $x  = \begin{bmatrix}1\\0\\-1\\0\\1\end{bmatrix}$

\textbf{\large Opgave 2.}\\

\textbf{(a)}\\

Vi har flg. 

\[
x_1\begin{bmatrix}
2 \\ 4 \\ 1\\ 5 \\1
\end{bmatrix} + 
x_2\begin{bmatrix}
1 \\ 2 \\ 1\\ 3 \\0
\end{bmatrix} + 
x_3\begin{bmatrix}
4 \\ 8 \\ 1\\ 9 \\6
\end{bmatrix} + 
 = \begin{bmatrix}
0\\0\\3\\3\\0
\end{bmatrix}
\] 

Totalmatricen sættes op og vi løser system

\[
\begin{bmatrix}[ccc|c]
2 & 1 & 4 & 0 \\
4 & 2 & 8 & 0 \\
1 & 1 & 1 & 0 \\
5 & 3 & 9 & 3 \\
1 & 0 & 6 & 0
\end{bmatrix} : (x_1, x_2, x_3) = (-6 , 8, 1)
\]

Vi laver nu en lineær kombination med konstanterne og tester om de giver $v_3$

\[
v_3 = -6 \begin{bmatrix} 2 \\ 4 \\ 1 \\ 5 \\ 1 \end{bmatrix} +
		  8 \begin{bmatrix} 1 \\ 2 \\ 1 \\ 3 \\ 0 \end{bmatrix} +
		    \begin{bmatrix} 4 \\ 8 \\ 1 \\ 9 \\6 \end{bmatrix} =
		\begin{bmatrix}
		0 \\ 0 \\ 3 \\ 3 \\ 0
		\end{bmatrix}
\]\\

ergo er koordinaterne $ (x_1, x_2, x_3) = (-6 , 8, 1)$ \\

\textbf{(b)}\\

Det opgives at $v_1 = u_2,  $ og $ v_2 = u_3$

Så gælder det at
$[v_1]_\beta = 0u_1  + 1u_2 + 0u_3 = \begin{bmatrix}0\\1\\0\end{bmatrix}$\\
$[v_2]_\beta = 0u_1  + 0u_2 + 1u_3 = \begin{bmatrix}0\\0\\1\end{bmatrix}$

Ergo har vi basisskiftet $[v]_{\mathcal{B}  \leftarrow  \mathcal{C} } =\begin{bmatrix}0 & 0 & -6\\1 & 0 & 8\\0 & 1 & 1\end{bmatrix} $


\textbf{c)}\\

\[w = v_1 + 2v_2 - v_3 \leftrightarrow 
\begin{bmatrix}
1 \\ 2\\1\\3\\0
\end{bmatrix} + 2
\begin{bmatrix}
4 \\ 8 \\ 1 \\ 9 \\ 6
\end{bmatrix} -
\begin{bmatrix}
0 \\0 \\3 \\3 \\0
\end{bmatrix}  =
\begin{bmatrix}
9 \\18 \\0 \\18 \\6
\end{bmatrix} 
\]

$w$ udtrykkes som en lineær kombination af $\beta$
\[
x_1 \begin{bmatrix} 2 \\ 4 \\ 1 \\ 5 \\ 1\end{bmatrix} +
x_2 \begin{bmatrix} 1 \\ 2 \\ 1 \\ 3 \\ 0\end{bmatrix} +
x_3 \begin{bmatrix} 4 \\ 8 \\ 1 \\ 9 \\ 6\end{bmatrix} =
\begin{bmatrix}
9 \\18 \\0 \\18 \\6
\end{bmatrix} 
\]

Vi udregner $x_1 .. x_3$ ved at løse ligningssystemet

\[
\begin{bmatrix}[ccc|c]
2 & 1 & 4 & 9 \\
4 & 2 & 8 & 18\\
1 & 1 & 1 & 0 \\
5 & 3 & 9 & 18 \\
1 & 0 & 6 & 6
\end{bmatrix} 
\]


Ligningsystemet løses og vi får: $(x_1, x_2, x_3) = (12, -11, -1)$ \\

ergo er lineærkombinationen 
\[
12 \begin{bmatrix} 2 \\ 4 \\ 1 \\ 5 \\ 1\end{bmatrix} 
-11 \begin{bmatrix} 1 \\ 2 \\ 1 \\ 3 \\ 0\end{bmatrix} 
-1 \begin{bmatrix} 4 \\ 8 \\ 1 \\ 9 \\ 6\end{bmatrix} =
\begin{bmatrix}
9 \\18 \\0 \\18 \\6
\end{bmatrix} 
\]

\textbf{\large Opgave 3.}\\

\textbf{a)}\\
\[
\left (
\begin{matrix}
c_1^f \\
c_2^f
\end{matrix}
\right )
=
\left (
\begin{matrix}
c_1\\
c_2
\end{matrix}
\right )
+
\vec {CS}
=
\left (
\begin{matrix}
c_1\\
c_2
\end{matrix}
\right )
+
\left (
\begin{matrix}
s_1 - c_1 \\
s_2 -c_2 
\end{matrix}
\right )
=
\left (
\begin{matrix}
s_1\\
s_2
\end{matrix}
\right )
\]

\[
\left (
\begin{matrix}
s_1^F\\
s_2^F
\end{matrix}
\right ) =
\left (
\begin{matrix}
s_1\\
s_2
\end{matrix}
\right ) + 
\left (
\begin{matrix}
s_1 - c_1\\
s_2 - c_2
\end{matrix}
\right ) = 
\left (
\begin{matrix}
2s_1 - c_1\\
2s_2 - c_2
\end{matrix}
\right ) 
\]

\textbf{b)}\\
$$
\begin{bmatrix}
c_1^F \\ c_2^F \\ s_1^F \\ s_2^F
\end{bmatrix} = F 
\begin{bmatrix}
c_1 \\ c_2 \\ s_1 \\ s_2
\end{bmatrix} = 
\begin{bmatrix}[cccc|c]
0 & 0 & 1 & 0 & c_1 \\ 
0 & 0 & 0 & 1 & c_2 \\ 
-1 & 0  & 2 & 0 & s_1 \\ 
0 & -1 & 0 & 2 & s_2
\end{bmatrix} 
\leftrightarrow F = 
\begin{bmatrix}
0 & 0 & 1 & 0 \\
0 & 0 & 0 & 1  \\ 
-1 & 0  & 2 &  0 \\ 
0 & -1 & 0 & 2 
\end{bmatrix} 
$$

\textbf{c)}\\

Fra $b$ kender vi $F$

Vi har fået oplyst detaljerne omkring $L_\theta$ og $R_\theta$\\


$L_\theta = 
	\begin{bmatrix} 
		1 & 0 & 0 & 0\\
		0 & 1 & 0 & 0 \\
		 0.133974 & 0,5 & 0,87 & -0,5\\
		-0,5 & -0,133974 & 0,5 & 0.87   
	\end{bmatrix}
R_\theta = 
	\begin{bmatrix} 
		1 & 0 & 0 & 0\\
		0 & 1 & 0 & 0 \\
		 0,133974 & -0,5 & 0,87 & 0,5\\
		0,5 & 0.133974 & -0,5 & 0.87   
	\end{bmatrix}
$\\

På matrixprogrammet udregnes $  F \cdot L_\theta \cdot F \cdot R_\theta \cdot R_\theta \cdot F  =$
\begin{verbatim}
F.mul( L ).mul( F ).mul( R ).mul( R ).mul( F ).mul( K );
[ 1,37 ]
[ 2,37 ]
[ 1,87 ]
[ 3,23 ]
\end{verbatim}


hvor K = $\begin{bmatrix}0\\0\\0\\1\end{bmatrix}$ (Startposition)


\textbf{\large Opgave 4.}\\

\textbf{a, b)}\\

Kig i java kode\\\\

\textbf{c}\\

Eksempel på kørsel, med input fra opgavebeskrivelse.\\

Matrix\\

[ 2,00 4,00 2,00 8,00 5,00 ]

[ 1,00 7,00 7,00 10,00 5,00 ]

[ 0,00 1,00 1,00 3,00 3,00 ]\\

Gauss\\

[ 0,00 1,00 1,00 3,00 3,00 ]

[ 1,00 0,00 -1,00 -2,00 -3,50 ]

[ 0,00 0,00 1,00 -9,00 -12,50 ]\\

GaussJordan\\

[ 0,00 1,00 0,00 12,00 15,50 ]

[ 1,00 0,00 0,00 -11,00 -16,00 ]

[ 0,00 0,00 1,00 -9,00 -12,50 ]


}
\end{document}

